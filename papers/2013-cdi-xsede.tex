\documentclass{sig-alternate}

\usepackage{url} 
\usepackage{subfigure} 
\usepackage{color}
\usepackage{graphicx}
\usepackage{xspace}
\usepackage{listings}


\usepackage[font=footnotesize]{subfig}
\DeclareCaptionType{copyrightbox}
\hyphenation{op-tical net-works semi-conduc-tor}

\newif\ifdraft
\drafttrue
\ifdraft
\newcommand{\tbnote}[1]{ {\textcolor{magenta} { ***TcB: #1 }}}
\newcommand{\jhanote}[1]{ {\textcolor{red} { ***SJ: #1 }}}
\newcommand{\mrnote}[1]{ {\textcolor{blue} { ***Melissa: #1 }}}
\newcommand{\abhi}[1]{ {\textcolor{green} { ***Abhinav: #1 }}}
\newcommand{\hfnote}[1]{ {\textcolor{cyan} { ***HF: #1 }}}
\newcommand{\rajib}[1]{ {\textcolor{blue} { ***RM: #1 }}}
\else
\newcommand{\rajib}[1]{}
\newcommand{\jhanote}[1]{}
%\newcommand{\note}[1]{ {\textcolor{blue} { ***NOTE: #1 }}}
\newcommand{\note}[1]{ {}}
\newcommand{\abhi}[1]{ {}}
\newcommand{\hfnote}[1]{}
\newcommand{\tbnote}[1]{ {}}
\fi

\newcommand{\pilotjob}{Pilot-Job\xspace}
\newcommand{\pilotjobs}{Pilot-Jobs\xspace}
\newcommand{\pilotdata}{Pilot-Data\xspace}
\newcommand{\pilotapi}{Pilot-API\xspace}

\begin{document}

\conferenceinfo{XSEDE13}{'13 }
 \CopyrightYear{2013}

 \title{A Framework for Flexible and Scalable Replica-Exchange on
   Production Distributed CI}

% \numberofauthors{5} 
% \author{
% \alignauthor
%  \\
%        \affaddr{x}\\
%        \affaddr{x}\\
%        \email{x}
% % 2nd. author
% \and  % use '\and' if you need 'another row' of author names
% % 3rd. author
% \alignauthor 
% B \\
%        \affaddr{xxxy}\\
%        \affaddr{xxx}\\
%        \email{xxx}
% % 5th. author
% \alignauthor 
% C \\
%        \affaddr{Rutgers University}\\
%        \affaddr{Piscataway, NJ 08854}\\
%        \email{xxx}
% }

\maketitle

\begin{abstract}

  A description of the computational workflow, viz., the different
  machines/resources used, how many ensembles we simulated, data
  volumnes managed etc., (ii) any computational performance issues
  including (a) measuring "efficiency" as the number of distributed
  resources utilized goes as measured by $T_c$ (time to completion),
  (b) $T_c$ as a function of the number of replicas, and (iii) a
  description of the software infrastructure that is employed to
  enable distributed replica-exchange on XSEDE.

\end{abstract}

\category{H.4}{Information Systems Applications}{Miscellaneous}
%A category including the fourth, optional field follows...
\category{D.2.8}{Software Engineering}{Metrics}[complexity measures, performance measures]

\terms{Experience, Technology}

\keywords{HPC, Distributed Computing, NAMD, MD, Large Scale, XSEDE
  resources}

\section{Introduction}
\jhanote{SJ's section}

\jhanote{Describe the need for large-scale simulations. Describe what
  is unique to replica-exchange. Challenges that arise from loose,
  intermittent (stochastic) coupling, as opposed to fixed exchanges}

\jhanote{(i) algorithmic advances coupled with infrastructural
  advances. (ii) the need for dynamic execution and resource
  management, (iii) which is why we believe pilot-jobs is a good
  abstraction (reference earlier work using BigJob). (iv) Novelty of
  this work is: (A) marriage of an application library for flexible
  replica-exchange simulations with scalable and dynamic resource
  management, (B) demonstration of a framework for different and
  tunable replica coupling schemes that can be used for QM and
  classical examples.}
 
\jhanote{Aim of this work is to present (i) the framework -- both
  components, (ii) understand and characterize basic performance}
  
\section{Scientific Problem}\label{sec:requirements}

\jhanote{RE represents the core coordination/communication pattern for
  a class of problems}
\mrnote{Chemists}

\subsection{Describing the Replica-Exchange ``Workflow'' } \label{}

\subsection{Computational Requirements}

\section{Software Environment}

XSEDE is inherently a very complex infrastructure. Given
the wide range of user groups and application use cases it
aims to address, and the large number and the diversity
of participating resource providers, this is to be expected.
Complex systems are though usually very difficult to use, as
that complexity and the underlying resource diversity often
translates into complicated user tools and interfaces. The
relatively clean architecture of XSEDE is, to some extent,
addressing this problem, but is, in itself and at this point in
time, a moving target.

In order to address this problem, we utilize Pilot-Jobs
to harness the power of the distributed, heterogeneous XSEDE
resources. A pilot-job is a mechanism by
which a proxy for the actual simulations is submitted on the
resource to be utilized; this proxy agent in turn, is given
responsibility to convey to the application the availabilty of
resources and also influence which tasks are executed. The
abstraction of a Pilot-Job generalizes the reoccurring concept
of utilizing a placeholder job as a container for a set of
compute tasks; instances of that placeholder job are commonly referred 
to as Pilot-Jobs or pilots. Analagous to Pilot-Jobs, Pilot-Data is used to seperate the allocation of physical storage and application-level data units. 

In general, Pilot-Abstractions provide a suitable means to marshal heterogeneous sets of both compute and data resources and support the efficient utilization of different kinds of commercial as well science cloud resources. Pilot-Abstraction have been extensively used on both HPC and HTC infrastructures for a range of application scenarios as a resource management abstraction to, (i) improve the utilization of resources, (ii) to reduce wait times of a collection of tasks, (iii) to facilitate bulk or high-throughput simulations where multiple jobs need to be submitted which would otherwise saturate the queuing system, and (iv) as a basis to implement application specific execution, scheduling and policy decisions

The P* model~\cite{pstar}, Pilot-Abstractions model, worked to clearly define the computation and data components of a distributed application as 'compute units' and 'data units' in the context of Pilot-Jobs and Pilot-Data. A compute unit describes a self-containing piece of work, e.g. a computational task that potentially operates on a set of input data, while a data unit is a container for a logical group of data that is often accessed together or comprises a larger set of data; e.g. a data file or chunk.


\subsection{BigJob: A Pilot-based Framework}

A specific implementation of the P* model is BigJob, which provides a framework for running many types of distributed applications -- including but not limited to very-large scale parallel simulations, many small high-throughput simulations, or ensemble-based workflows. Consistent with the P* model, BigJob (BJ) and the BigData (BD) extension provide a unified runtime environment for Pilot-Jobs and Pilot-Data on heterogeneous infrastructures. For this purpose, BigJob provides a higher-level, unifying interface to heterogeneous and/or distributed data and compute resources. The framework is accessed via the Pilot-API, which provides two key abstractions: Pilot-Job and Pilot-Data.

Applications can specify their resource requirements using a Pilot description. In the compute case, the user typically specifies the application to run as well as the number of cores required by their application. A Pilot-Data is defined using a service URL containing the reference to a storage resource. Pilots are started via the Pilot-Compute Service respectively the Pilot-Data Service.

BigJob eliminates the need to interact with different kinds of compute resources, e. g. batch-style HPC/HTC resources as well as cloud resources, and provide a unified abstraction for allocating resources. Similarly, BigData removes the need to interoperate with different data sources and stores, e. g. object stores, parallel filesystems, repositories, databases etc., by providing a virtualized data service layer, which can be used to allocated and access a heterogeneous set of data sources and stores.

BigJob has seen its widest usage across the heterogeneous resources that XSEDE provides. Simple installation into user space on any resource that supports Python 2.5 or greater makes the uptake of BigJob easy for the end user. BigJob supports thousand of jobs (millions of SUs) and has been at the heart of two recent and successful ECSS projects. BigJob has historically been used for parameter sweeps, many instances of the same task (ensemble), chained tasks, loosely coupled but distinct tasks, as well as tasks with data or compute dependencies.

% Note much more to come on BigJob

\subsubsection{SAGA: Interoperability Layer}


In order for BigJob to work on heterogeneous resources, it requires
an interoperability layer which provides access to a variety of middleware.
The Simple API for Grid Applications (SAGA) provides a well 
defined and stable API, which exposes those operations which
are required on application level, but encapsulates the complexity 
of translating them into the respective operations on
the XSEDE infrastructure. In other words: SAGA tries to
move the complexity of dealing with distributed cyberinfrastructures 
like XSEDE out of the application, and into the 
SAGA implementation layer, while providing the semantics 
necessary to efficiently implement distributed applications 
which can utilize XSEDE.

SAGA is an implementation of an Open Grid Forum (OGF) Technical 
Specification that provides a common and consistent high-level 
API for the most commonly required functionality to construct 
distributed applications. It also provides a high-level API to 
construct tools and frameworks to support distributed applications. 
The functional areas that are supported by SAGA include job-submission,
file transfer and access, as well as support for data streaming and distributed 
coordination. SAGA provides both a syntactic and semantic unification via a 
single interface to access multiple, semantically distinct middleware distributions.

The key advantages of the development using SAGA include, but 
are not limited to: i) to provide a general-purpose, commonly used yet standardized functionality, while hiding complexity of heterogeneity of back-end resources, 
ii) to provide building blocks for constructing higher-level functionality
and abstractions, iii) to provide the means for developing
broad range of distributed applications such as gateways,
workflows, application management systems, and runtime
environments. Interestingly, SAGA provides an integrated,
lightweight approach to support scripting for building distributed applications. The SAGA
API has been used to provide almost complete coverage over nearly all grid
and distributed computing middleware/systems, including
but not limited to Condor, Genesis, Globus, UNICORE, SGE,
LSF/PBS/Torque, and Amazon EC2.

% Andre M to give more of a section for SAGA

\subsubsection{Deployment into User Space}

SAGA and BigJob are lightweight that they can be easily installed into 
the home directory of a user using the Python Package Index (PyPi). SAGA is 
packaged within BigJob, so users do not have to worry about installing two
seperate modules. 

Since the main deployment is on XSEDE, we do not recommend
altering the default PYTHONPATH of the machines. Instead, we encourage users to
use a virtual environment. A virtual environment allows a user to create a local
Python software repository in his or her home directory that behaves exactly like the 
global Python repository, except that it grants the user \textit{write} access to it. In 
order to use virtual environment, the Python version must be greater than 2.4. Since 
some XSEDE machines use Python 2.4 as the default python version, it may be required
to load a python module file before installing BigJob.

After activating the virtual environment, the BigJob python package can be installed
by typing: 

\begin{lstlisting}[frame=single]
easy_install bigjob
\end{lstlisting}

In addition to the BigJob package, it installs BigJob python
dependencies, including the SAGA package. The SAGA package includes the proper
adaptors for a wide variety of middleware systems. This allows
the user to submit jobs to almost any of the XSEDE batch queuing systems.

BigJob requires SSH password-less login to the machines and a redis server running either locally or on a remote server. The redis server is used for coordination
of the pilot-job and its compute and data units. For the purposes of this project, we utilize a private redis server hosted on a virtual machine at Indiana University. In order to provide a more seemless uptake of BigJob by users, we will provide an open-acess redis server available on XSEDE. This avoids the overhead of new users having to start a redis-server on an XSEDE machine's head node or on their local machines. This effort is currently underway with XSEDE ECSS staff to make the server only accessible to registered users of XSEDE.

After following the aforementioned steps, users will be able to write their own BigJob submission scripts using Python. These scripts can range from simple ensembled-based simulations to more complicated workflows based on the users' needs.

\subsection{Async Replica-Exchange}
\mrnote{CompSci + Chemists}

\section{Characterizing Replica Exchange on XSEDE }\label{sec:results}
\mrnote{ CompSci's Section}

\subsection{Performance Model}

\jhanote{Ole}

\subsection{Systems Investigated} 

(i) Systems 1 -3 , (ii) $N_R$ vary from 4, 16, 32, 64, 128, 512, (iii)
$T_c = T_W + T_Q$ where terms are total time to completion, run time,
time waiting in queue.  (iv) each exchange takes place approximately
1ps (v) each replica runs for 1ns


\subsection{Computational Configuration}
(i) Fixed num. of cores [Fixed Pilots, varying resources] (ii) ``Flood"
XSEDE -- minimize the Time to Wait

\section{Discussion}

\jhanote{Overall Productivity is Scientific efficiency vs. Computing
  efficiency!}

\subsection{Experience}

\subsubsection{Issue of Resiliency/Redundacy} Emilio has interesting
antidotes -- nodes crashing and RE starts limping but as nodes got
rebooted, started speeding up.  Experience --> Old to New Aspects
\texttt{get\_state()} - State monitoring capability


\subsection{Future Work}

\subsubsection{Multiple pilots}

% \subsubsection{Local}
\subsubsection{Logically Distributed}
\subsubsection{Physically Distributed}

\subsubsection{Different Exchange Modes}


\subsection{Conclusion}



\section*{Acknowledgement}
\footnotesize{Additional important
  funding has been provided by NSF-ExTENCI (OCI-1007115) Important
  funding for SAGA has been provided by the UK EPSRC grant number
  GR/D0766171/1 (via OMII-UK) and HPCOPS NSF-OCI 071087. Bishop were
  supported by NIH-R01GM076356.}

\bibliographystyle{abbrv}
%\bibliography{saga,saga-old,literature,replica-exchange}
%\bibliography{saga2,literature,replica-exchange}

\end{document}
